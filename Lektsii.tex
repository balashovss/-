\documentclass[12pt]{article}
\usepackage{amsmath}
\usepackage{graphicx}
\usepackage{hyperref}
\usepackage[utf8x]{inputenc}
\usepackage[T1]{fontenc}
\usepackage[russian]{babel}

\newtheorem{definition}{Опр.}

\title{Защита интеллектуальной собственности и правоведение}

\author{Балашов Семён // Елисеев Никита // Ермошин Максим}

\date{2024–09–18}

\begin{document}

\maketitle

\section{Лекция 1}

\dots

\section{Лекция 2}

ИС(инт. собственность) - результаты интеллектуальной деятельности и приравнивание к ним средства индивидуальной юридических лиц, товаров, работ, услуг и предприятий
Объекты ИС: результаты инт. деятельности(РИД) - нормальные результаты труда, средства индивидуализации(СИ) - чисто эмблемки всякие и названия
	
РИД: объекты авторских и смежных прав(произв. науки и искусства, программы для ЭВМ, базы данных), объекты патентных прав(изобретения), объекты особых прав(селекционные достижения, топология печатных плат и секреты производства)
	Авторские права и смежные права(не требуют обязательной гос регистрации, охрана возникает по факту воплощения в материальной форме)
Патентные права(обязательная государственная регистрация, охрана предоставляется на основании охранного документа)
Авторское и патентное отличаются также сроком действия

Субъекты инт. собственности: автор(физ. лицо, творческим трудом которого создан результат инт. деятельности), правообладатель(гражданин или юр. лицо, обладающие исключительным правом на результат инт. деятельности или средство индивидуализации)

Виды инт. прав: личные неимущественные(право авторства), исключительное право(использование и распоряжение), иные права.

Для автора все виды прав. Личные неим. - это про то что оно не переходит наследникам, исключительное право на сто процентов принадлежит всем его собственникам(можно продать, подарить и передать по наследству), иные свойственны конкретной группе.

Личные неимущественные - бессрочно действует.

Для каждого отдельного результата интеллектуальной деятельности свои сроки на исключительные права, установленные законодательством. Причем начинается этот срок с момента подачи заявки.
	Формализация режима правовой охраны: депонирование(авторское свидетельство), гос. регистрация(обязательная или по желанию правообладателя)
	Для гос. регистрации выдается либо свидетельство(программы ЭВМ, топология микросхем, товарные знаки и знаки обслуживания) либо патент(изобретения и селекционные достижения действует только на территории отдельно взятой страны)(свидетельство рейтят меньше патентов)
	РФ является участницей 24 международных конвенций из 25
	Защита ИС - это совокупность юридически значимых действий по защите нарушенных или оспариваемых прав...

\section{Лекция 3. Авторское право}

Гражданский кодекс РФ Часть 4 Гл 70

Постановление Пленума Верховного суда РФ 

\begin{definition}
Авторское право  - интеллектаульное право на произведения 
науки, литературы и искусства.
\end{definition}

\begin{definition}
Авторское право - совокупность предоставленных авторских прав 
(полномочий), которые возникают в связи с созданием 
произведений и их использования обществом.
\end{definition}

\begin{definition}
Авторское право - правовое положение авторов и созданных 
творческим трудом произведений науки, литературы и искусства.
\end{definition}

\begin{definition}
Авторское право - предоставленное законом исключительное 
право автора произведения огласить себя творцом этого 
произведения, воспроизводить его, распространяться или 
доводить до сведения публики любыми способами и средствами, 
а также разрешать другим пользоваться произведением на 
законных основаниях.
\end{definition}

Принципы правового регулирования 
(Бернская конвенция об охране 1886 г.):
\begin{itemize}
\item территориальный
\item свободны творчества
\item объединения личных и имущественных интересов
\item национальный
\item морального и материального стимулирования
\item охраны прав
\end{itemize}

Особенности правовой охраны
\begin{itemize}
\item охр возн по праву создания в форме, которая может быть воспринята и воспроизведена
\item охране подлежит форма, в которую воплащена идея
\item не требуется государственная регистрация или иные формальности
\item государственная регистрация программ ЭВМ и баз данных - по желанию правообладателя
\end{itemize}

Интеллектуальная собственность
\begin{itemize}
\item  РИД

произведения науки лит искусства
ЭВМ
Базы данных

\item СИ

фирменные наименовения ...
\end{itemize}

Объекты авторских прав
Произведения литературы и искусства это:
\begin{itemize}
\item литературные произведения
\item драмат и муз-драмат произв, сценарные произв.
... (ст. 1259 ГК РФ)
\end{itemize}

Производное произведение (ст. 1260, 1263 ГК РФ)
произведение, которое представляет собой переработку другого произведения (перевод: аранжировка, экранизация)

\begin{definition}
Составное произведение - произведение, по подбору или расположению материалов, результат творческого труда
\end{definition}

аудиовизуальное произведение - произведение, состоянее из зафиксированное серии связанных межжду собой изображений 
(с сопровождением или без сопровождения звуком) и предназначенное для хрительного или слухового восприятия с
помощью соотв устройств

Программа для ЭВМ
ст 1261 ГК РФ

База данных 
ст 1260, 1334 ГК РФ

Смежные права VS авторское право ст 1260,, 1240 ГК РФ

Охрана предоставляется вне зависимости от:
\begin{itemize}
\item достоинства
\item назначения
\item факта обнародования
\item степени завершенности произведения
\item соблюдения формальностей
\end{itemize}

Обнародованное (опубликованное) произведение
Необнародованное произведение

Формализация режима правовой охраны
\begin{itemize}
\item Депонирования
\item Государственная регистрация
\item Депонирование может установить \dots
\end{itemize}

Содержание произведения - это идеи и принципы, которые положены в основу
произведения. Структура самого содержания зависит от вида произведения.

Примеры:
\begin{itemize}
\item в научной статье
\item в архитектурном проекте
\item в фотографии
\end{itemize}

Форма выражения произведений

Форма произведения
\begin{itemize}
\item внутренняя
\item внешняя
\end{itemize}

Не охраняются авторским правом:
\begin{itemize}
\item идеи, концепции, принципы
\item \dots
\end{itemize}

Самостоятельные произведения

Производные произведения
\begin{itemize}
\item авторские права на осуществленную переработку
\item охраняются как Самостоятельные
\item реализуют свои права при условии соблюдения прав автора
основного произведения
\item авторы составных произведений являются самостоятельными
и независимыми от составного произведения
\end{itemize}

\begin{definition}
Соавторы (ст. 1259 ГК РФ) - граждане, создавшие произведение 
совместным трудом. Независимо от того, образует ли такое произведение
неразрывное целое или состоит из частей, каждая из которых имеет 
самостоятельное значение
\end{definition}

Произведение, созданное в соавторстве, используется \dots

Виды интеллектуальных прав
\begin{itemize}
    \item личные неимущественные
    \begin{itemize}
    \item авторство
    \item автора ан имя
    \item на неприкосновенность
    \item на обнародование
    \end{itemize}
    \item исключительное
    \begin{itemize}
    \item использование
    \item распоряжение
    \end{itemize}   
    \item иные права
    \begin{itemize}
    \item право доступа
    \item право следования
    \item право на отзыв
    \item право на вознаграждение
\end{itemize}   
\end{itemize}

Срок действия исключительного права (ст. 1256 ГК РФ)
в течение всей жизни автора и 70 лет, считая с 1 
января года, следующего за годом смерти автора.

Созданное в соавторстве - в течение всей жизни автора, 
пережившего других соавторов, и 70 лет, исчтая с 1 января года,
следующего за годом смерти автора.

Анонимно - через 70 лет с 1 января следующего года за годом обнародования.

Обнародованное после смерти - через 70 лет, считая 
с 1 января следующего года за годом обнародования.
\end{document}