\documentclass[12pt]{article}
\usepackage{amsmath}
\usepackage{graphicx}
\usepackage{hyperref}
\usepackage[utf8x]{inputenc}
\usepackage[T1]{fontenc}
\usepackage[russian]{babel}
\usepackage[a4paper, total={7in, 9.5in}]{geometry}
\newtheorem{definition}{Опр.}

\title{Защита интеллектуальной собственности и правоведение}

\author{Ермошин Максим}

\date{2024–09–18}

\begin{document}

\maketitle

\section{Лекция 1}

\dots

\section{Лекция 2}

\begin{definition}
ИС(инт. собственность) - результаты интеллектуальной деятельности и приравнивание к ним средства индивидуальной юридических лиц, товаров, работ, услуг и предприятий
\end{definition}

\begin{definition}
Объекты ИС: результаты инт. деятельности(РИД) - нормальные результаты труда, средства индивидуализации(СИ) - чисто эмблемки всякие и названия
\end{definition}

РИД: 
\begin{itemize}
\item объекты авторских и смежных прав(произв. науки и искусства, программы для ЭВМ, базы данных)
\item объекты патентных прав(изобретения), 
\item объекты особых прав(селекционные достижения, топология печатных плат и секреты производства)
\end{itemize}

	Авторские права и смежные права(не требуют обязательной гос регистрации, охрана возникает по факту воплощения в материальной форме)
Патентные права(обязательная государственная регистрация, охрана предоставляется на основании охранного документа)
Авторское и патентное отличаются также сроком действия

Субъекты инт. собственности: 
\begin{itemize}
\item автор(физ. лицо, творческим трудом которого создан результат инт. деятельности), 
\item правообладатель(гражданин или юр. лицо, обладающие исключительным правом на результат инт. деятельности или средство индивидуализации)
\end{itemize}

Виды инт. прав: 
\begin{itemize}
\item личные неимущественные(право авторства), 
\item исключительное право(использование и распоряжение), 
\item иные права.
\end{itemize}

Для автора все виды прав. Личные неим. - это про то что оно не переходит наследникам, исключительное право на сто процентов принадлежит всем его собственникам(можно продать, подарить и передать по наследству), иные свойственны конкретной группе.

Личные неимущественные - бессрочно действует.

Для каждого отдельного результата интеллектуальной деятельности свои сроки на исключительные права, установленные законодательством. Причем начинается этот срок с момента подачи заявки.
	Формализация режима правовой охраны: депонирование(авторское свидетельство), гос. регистрация(обязательная или по желанию правообладателя)
	Для гос. регистрации выдается либо свидетельство(программы ЭВМ, топология микросхем, товарные знаки и знаки обслуживания) либо патент(изобретения и селекционные достижения действует только на территории отдельно взятой страны)(свидетельство рейтят меньше патентов)
	РФ является участницей 24 международных конвенций из 25
	Защита ИС - это совокупность юридически значимых действий по защите нарушенных или оспариваемых прав...

\section{Лекция 3. Авторское право}

Гражданский кодекс РФ Часть 4 Гл 70

Постановление Пленума Верховного суда РФ 

\begin{definition}
Авторское право  - интеллектуальное право на произведения 
науки, литературы и искусства.
\end{definition}

\begin{definition}
Авторское право - совокупность предоставленных авторских прав 
(полномочий), которые возникают в связи с созданием 
произведений и их использования обществом.
\end{definition}

\begin{definition}
Авторское право - правовое положение авторов и созданных 
творческим трудом произведений науки, литературы и искусства.
\end{definition}

\begin{definition}
Авторское право - предоставленное законом исключительное 
право автора произведения огласить себя творцом этого 
произведения, воспроизводить его, распространяться или 
доводить до сведения публики любыми способами и средствами, 
а также разрешать другим пользоваться произведением на 
законных основаниях.
\end{definition}

Принципы правового регулирования 
(Бернская конвенция об охране 1886 г.):
\begin{itemize}
\item территориальный
\item свободны творчества
\item объединения личных и имущественных интересов
\item национальный
\item морального и материального стимулирования
\item охраны прав
\end{itemize}

Особенности правовой охраны
\begin{itemize}
\item охрана возникает по праву создания в форме, которая может быть воспринята и воспроизведена
\item охране подлежит форма, в которую воплащена идея
\item не требуется государственная регистрация или иные формальности
\item государственная регистрация программ ЭВМ и баз данных - по желанию правообладателя
\end{itemize}

Интеллектуальная собственность
\begin{itemize}
\item  РИД
\begin{itemize}
    \item  произведения науки лит искусства
    \item  ЭВМ
    \item  Базы данных
\end{itemize}
\item СИ
\begin{itemize}
\item  фирменные наименовения 
\item ...
\end{itemize}
\end{itemize}

Объекты авторских прав

Произведения литературы и искусства это:
\begin{itemize}
\item литературные произведения
\item драмат и муз-драмат произв, сценарные произв.
... (ст. 1259 ГК РФ)
\end{itemize}

\begin{definition}
Производное произведение (ст. 1260, 1263 ГК РФ)
произведение, которое представляет собой переработку другого произведения (перевод: аранжировка, экранизация)
\end{definition}

\begin{definition}
Составное произведение - произведение, по подбору или расположению материалов, результат творческого труда
\end{definition}
\begin{definition}
аудиовизуальное произведение - произведение, состоянее из зафиксированное серии связанных межжду собой изображений 
(с сопровождением или без сопровождения звуком) и предназначенное для хрительного или слухового восприятия с
помощью соотв устройств
\end{definition}

\begin{tabular}{ c c }
Программа для ЭВМ & ст 1261 ГК РФ \\ 
База данных  & ст 1260, 1334 ГК РФ\\  
Смежные права VS авторское право &  ст 1260,, 1240 ГК РФ    
\end{tabular}


Охрана предоставляется вне зависимости от:
\begin{itemize}
\item достоинства
\item назначения
\item факта обнародования
\item степени завершенности произведения
\item соблюдения формальностей
\end{itemize}

Обнародованное (опубликованное) произведение

Необнародованное произведение

Формализация режима правовой охраны
\begin{itemize}
\item Депонирования
\item Государственная регистрация
\item Депонирование может установить \dots
\end{itemize}

\begin{definition}
Содержание произведения - это идеи и принципы, которые положены в основу
произведения. Структура самого содержания зависит от вида произведения.
\end{definition}

Примеры:
\begin{itemize}
\item в научной статье
\item в архитектурном проекте
\item в фотографии
\end{itemize}

Форма выражения произведений

Форма произведения
\begin{itemize}
\item внутренняя
\item внешняя
\end{itemize}

Не охраняются авторским правом:
\begin{itemize}
\item идеи, концепции, принципы
\item \dots
\end{itemize}

Самостоятельные произведения

Производные произведения
\begin{itemize}
\item авторские права на осуществленную переработку
\item охраняются как Самостоятельные
\item реализуют свои права при условии соблюдения прав автора
основного произведения
\item авторы составных произведений являются самостоятельными
и независимыми от составного произведения
\end{itemize}

\begin{definition}
Соавторы (ст. 1259 ГК РФ) - граждане, создавшие произведение 
совместным трудом. Независимо от того, образует ли такое произведение
неразрывное целое или состоит из частей, каждая из которых имеет 
самостоятельное значение
\end{definition}

Произведение, созданное в соавторстве, используется используется соавторами совместно, если соглашением между ними не предусмотрено иное. В случае, когда такое произведение образует неразрывное целое, ни один из соавторов не вправе без достаточных оснований запретить использование такого произведения.

Виды интеллектуальных прав:
\begin{itemize}
    \item личные неимущественные
    \begin{itemize}
    \item авторство
    \item автора ан имя
    \item на неприкосновенность
    \item на обнародование
    \end{itemize}
    \item исключительное
    \begin{itemize}
    \item использование
    \item распоряжение
    \end{itemize}   
    \item иные права
    \begin{itemize}
    \item право доступа
    \item право следования
    \item право на отзыв
    \item право на вознаграждение
\end{itemize}   
\end{itemize}
\section{Лекция 4}
Срок действия исключительного права (ст. 1256 ГК РФ)
в течение всей жизни автора и 70 лет, считая с 1 
января года, следующего за годом смерти автора.

Созданное в соавторстве - в течение всей жизни автора, 
пережившего других соавторов, и 70 лет, исчтая с 1 января года,
следующего за годом смерти автора.

Анонимно - через 70 лет с 1 января следующего года за годом обнародования.

Обнародованное после смерти - через 70 лет, считая 
с 1 января следующего года за годом обнародования.
\hypertarget{ux441ux432ux43eux431ux43eux434ux43dux43eux435-ux438ux441ux43fux43eux43bux44cux437ux43eux432ux430ux43dux438ux435-ux43fux440ux43eux438ux437ux432ux435ux434ux435ux43dux438ux44f}{%
\subsection{Свободное использование
произведения}\label{ux441ux432ux43eux431ux43eux434ux43dux43eux435-ux438ux441ux43fux43eux43bux44cux437ux43eux432ux430ux43dux438ux435-ux43fux440ux43eux438ux437ux432ux435ux434ux435ux43dux438ux44f}}

\begin{itemize}
\item
  В личных целях (не для исполнения трудовых функций)
\item
  В информационных, научных, учебных или культурных целях (объекты тоже
  используются не для коммерческих нужд)
\item
  Библиотеками, архивами и образовательными организациями
\end{itemize}

\textbf{Без согласия} автора или иного правообладателя и \textbf{без
выплаты} вознаграждения, но с обязательным указанием имени автора,
произведение которого используется, и источника заимствования. Программы
для ЭВМ регистрируются в роспатенет (депонирование) Можно использовать
без согласия автора в определенных ситуациях (см выше) Как брать в
интеренете контент бесплатно?

\begin{itemize}
\item
  Условно бесплатные стоки (в соглашении указано, что платно, что
  бесплатно и на каких основаниях)
\item
  Открытые лицензии (сильный-слабый копилефт. Сильный - как взяли так и
  отдали (если взяли бесплатно, то и отдали бесплатно, например код с
  гитхаба))
\item
  Машиночитаемые лицензии (для личного пользования, дял некоммерческого
  пользования, проч (BY; BY SA; BY NC; BY NC SA; BY ND; BY NC ND))
\item
  Создание пародий и картикатур
\end{itemize}

!! Помним о цитировании 
\subsection{Патентное право}
Интеллектуальные права на
изобренеия, полезные модели и промышленные образцы (ст 1345 ГК РФ)
Особенности правовой озраны объектов патентных прав 
\begin{itemize}
\item Охране возникаетт
по факту государственной регистрации 
\item Озхране подлежит форма и/или
содержание, в которыю воплощена идея 
\item Обязательная государственная
регистрации или иные формальности
\end{itemize}

Требования к охраняемым РИД отсутств в законе
Предъявляется законом
Виды интеллектуальных прав различаются
Основания возникновения прав различаются
Срок и территория действия исключительного права различаются

Патент - правоустанавливающий документ. Он удостоверяет:
\begin{itemize} 
\item приоритет изобретения/полезной модели/промышленного образца 
\item авторство изобретения/полезной модели/промышленного образца
\item исключительное право
\end{itemize}

\hypertarget{ux447ux442ux43e-ux43eux445ux440ux430ux43dux44fux44eux442-ux43fux430ux442ux435ux43dux442ux43dux44bux435-ux43fux440ux430ux432ux430}{%
\subsubsection{Что охраняют патентные
права?}\label{ux447ux442ux43e-ux43eux445ux440ux430ux43dux44fux44eux442-ux43fux430ux442ux435ux43dux442ux43dux44bux435-ux43fux440ux430ux432ux430}}

\begin{itemize}
\item
  Изобретения. В качестве изобретения охраняется \textbf{техническое
  решение} в любой област, относящееся к \textbf{продукту} (в частности,
  устройству, веществу, штамму микроорганизма, культуре клетор растений
  или животных) или \textbf{способу}
\item
  полезные модели. Продукт (устройство, комплексы, вещество, штамм
  микроорганизма, культура клеток растений или животных); способ
  (процесс осуществления действий над материальным объектом);
  ``изобретение на применение'' (применение ранее известного прдукта или
  способа по новому назначению)
\item
  Промышленные образцы
\end{itemize}

Открытие; научные теории и математические методы; решения внешнего вида;
правила игр; программы для ЭВМ; решения, заключающее только в
предоставлении информации не охраняется - \textbf{НЕ ОХРАНЯЮТСЯ}

Полезная модель - техническое решение, относящееся к устройству
Изобретение:
\begin{itemize}
\item Продукт - устройства - Комплексы и комплекты - Вещества -
Штаммы микроорганизмов - и т д \ldots{}
\end{itemize}

Промышленный образец - решение внешнего вида изделия промышленного или
кустарно-ремесленного производства

Каким условиям должны соответствовать изобретение, полезная модель и
промышленный образец, чтобы получить правовую охрану? 
\begin{itemize}
\item изобретение -
новизна; промышленная применимость; изобретательский уровень 
\item полезная
модель - овизна; промышленная применимость 
\item промышленный образец -
новизна; оригинальность
\end{itemize}

Изобретение является новым, если оно не известно из уровня техники.
Уровень техники для изобретения включает любые сведения, ставшие
обзедоступными в мире до даты приоритета изобретения. Изобретение имеет
изобретательский уровень, если для специалиста оно явным образом не
следует из уровня техники Изобретение является промышленно применимым,
если оно может быть использовано в промышленности, сельском хозяйстве,
здравоохранении, других отраслях экономики или в социальной сфере.

У промышленного образца полезность

Виды интеллектулальных прав: 
\begin{itemize}
  \item Личные неимущественные. Право авторства 
  \item Иные права. Право на получение патента. Право на получение
вознаграждения. Право на преждепользования. Право послепользования. 
\item ИСключительное право 
\item Ввоз на территорию РФ 
\item Изготовление 
\item Применение
\item Предложение о продаже 
\item Иное введение в гражданский оборот или
хранение для этих целей продукта, в котором использованы изобретение или
полезная модель, либо изделия, в котором использован промышленный
образец Трерритория действия исключительного права 
\item на основании патентов РФ - территория РФ 
\item При наличии зарубежных патентов и патентов
РФ - территория РФ и стран, выдавших патент
\end{itemize}

Срок действия:
\begin{itemize}
\item Личные неимущественные - бессрочн 
\item Исключительное право - 20(25 - для агрохимикатов и пестицидов) лет с даты подачи заявки -
изобретение; полезная модель - 10; промышленный образец 5(25) лет с даты
подачи заявки. Далее переход в общественное достояние
\end{itemize}

\hypertarget{ux43fux440ux430ux432ux43e-ux43dux430-ux43fux43eux43bux447ux435ux43dux438ux435-ux43fux430ux442ux435ux43dux442ux430}{%
\paragraph{Право на полчение
патента}\label{ux43fux440ux430ux432ux43e-ux43dux430-ux43fux43eux43bux447ux435ux43dux438ux435-ux43fux430ux442ux435ux43dux442ux430}}

\begin{itemize}
\item
  Первоначально принадлежит автору
\item
  Может быть передано или перейти к другому лицу по основаниям,
  установленным законом
\item
  Исключительно право на служебное изобретение, служебную полезную
  модель или служебный промышленный образец и право на получение патента
  принадлежит работодателю
\end{itemize}
\subsection{Право на вознаграждение} В течение 6
  мес со дня уведомления работодателя работником (ст 1370 ГК РФ)
  \subsection{Служеюные ИЗ, ПМ и ПО}

\begin{itemize}
\item
  автор. право авторства, исключительное право, право на получение
  вознаграждения, право на получение патента
\item
  работодатель. Исключительное право, право на получение патента
\end{itemize}


\section{Лекция 5}
Право на вознаграждение(есть только у физ. лица - автора) за служебные изобретения, полезные модели, производственные образцы.(регулируется 1370 ГК РФ)
Платят его не часто, так как вознаграждение выплачивает работодатель. Выплата авторского вознаграждения регулируется постановлением правительства РФ №1848.
Инициативный РИД 
\begin {itemize} 
\item 
договор отчуждения исключительного права
\item 
вхождение в рабочую группу
\end{itemize}
Преждепользование - объект создается третьими лицами, которые не подавали заявку на патент и они могут использовать его для своего производства без отчислений
(*Патент на изобретение действует 20 лет начинаем платить через пять лет)
Послепользование - если не заплатить за патент, то через два года после неоплаты патент анулируется и им можно пользоваться без роялти до того как не восстановится патент.
Процедура получения патента
\begin{itemize}
    \item
    Заявка на выдачу патента(инженер делает, оплачивает пошлину и оформляет документы, приоритетом заявки называется момент подачи заявки)
    \item
    Формальная экспертиза(проверка документов, проверка оплаты пошлин, публикация сведений о заявке, переход на следующий этап стоит денег и называется ходатайством)
    \item
    Экспертиза по существу(проверка условий патентоспособности, информационный поиск, запросы, изменение заявки, переход на следующий этап - оплата пошлины)
    \item
    Публикация и выдача патента -  на этом этапе в заявку уже нельзя внести изменения
\end{itemize}
Служебные объекты(РИД)
Служебные объекты создаются по инициативе работодателя. Право на подачу патента отдается тоже ему и право на использование также.
Служебные РИД\begin{itemize}
    \item трудовой договор
    \item служебное задание(может быть в любой форме)
    \item уведомление о создании РИД(может быть в любой форме)
\end{itemize}
\subsection{Интеллектуальная собственность}
\begin{itemize}
    \item РИД \begin{itemize}
        \item произведения науки
        \item программы ЭВМ
        \item Базы данных
    \end{itemize}
    \item Средства индивидуализации
\end{itemize}
Источники норм -  ФЗ о коммерческой тайне
Секрет производства - это любая информация о результатах интеллектуальной собственности.
Секрет производства \begin{itemize}
    \item необщеизвестность,
    \item необщедоступность,
    \item оборотоспособность
\end{itemize}  
Секреты производства могут быть в любой сфере. 
Коммерческая тайна - это режим в который помещается секрет производства(объект интеллектуального права)
Секрет производства - сведения любого характера о РИД и о способах осуществления проффесиональной деятельности
Статья 5 ФЗ о коммерческой тайне. Пункт 4 статьи ФЗ об информации определяет информацию которую нельзя засекречивать.(нормативным правовым актам, информации о состоянии окружающей среды, о деятельности гос. органов, накапливаемой в открытых фондах библиотек, содержащейся в архивных фондов)
У обладателя сведений свобода выбора для охраны конфиденциальности 
\begin{itemize}
    \item Определение перечня информации
    \item Ограничение доступа к информации путем установления порядка обращения с этой информацией
    \item Учет лиц получивших доступ к информации
    \item Регулирование отношений по использованию информации
    \item Нанесение на материальные носители, грифа Коммерческая тайна с указанием обладателя такой информации
\end{itemize}
На секрет производства распространяется только исключительное право
Исключительное право возникает в момент принятия мер охраны сведений прекращает при утрате конфиденциальности либо при утрате коммерческой привлекательности
Разглашение коммерческой тайны \begin{itemize}
    \item дисциплинарная ответственность
    \item уголовная(большие убытки и удалось доказать причинно-следственную связь)
    \item гражданско-правовая
\end{itemize}
\end{document}